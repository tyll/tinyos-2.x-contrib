% keywords: Wireless Sensor Networks, Time Synchronization
The need for time synchronization in Wireless Sensor Networks  (WSNs) is critical for accurate timestamping of events and coordination of wake and sleep duty cycles.
% keywords: Precision Requirements
Typical time synchronization protocols for WSNs tend to deliver a time precision that in many times is higher than required, thus, in such times, these protocols tend to waste vast amounts of valuable resources trying to achieve a time precision that is not required, either, by its applications or Medium Access Control (MAC) layer protocols.
% keywords: Cross-Layer, Adaptive Time Synchronization
We here, purpose Tagus Time Synchronization Protocol (TTSP), a time synchronization protocol with a cross-layered architecture, that adaptively delivers the required time precision.
% keywords: Tagus-SensorNet
The proposed protocol was implemented in TinyOS for the Crossbow micaZ mote platform and deployed in the Tagus-SensorNet, a WSN testbed in IST-TUL Taguspark campus main building.