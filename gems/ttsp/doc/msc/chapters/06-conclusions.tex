\chapter{Conclusions}

%-----------------------------------------------------------
% keywords: Time synchronization in WSN, classic protocols
%-----------------------------------------------------------
The need for time synchronization in WSNs is critical for data consistency, MAC protocols and ultimately extending the network lifetime with coordinated sleeping and wake up cycles. Classic time synchronization protocols found on wired and common wireless networks, do not fill the necessary requirements for being deployed on a WSN.\\

%-----------------------------------------------------------
% keywords: Available time synchronization protocols for WSNs
%-----------------------------------------------------------
However, there has been effort on developing time synchronization protocols that do try to meet the requirements of time synchronization on WSNs. These protocols can be broadly classified by their approaches to pair-wise and network-wide synchronization. Common approaches found for pair-wise synchronization are implemented by a scheme of Sender-to-receiver or a Receiver-to-receiver. While for network-wide synchronization common approaches used are a Master-slave or a Peer-to-peer scheme. These classifications broadly classify a time synchronization protocol, although there are many other aspects to take into account. One common metric of performance used by many authors to evaluate their own protocol, its the time precision error obtained. It is evident that, lower time precision error obtained, more time precision is given to the applications operating on the WSN. But most of the times we might be spending a lot of effort on delivering a time precision that is not needed by the application itself. This effort is intrinsically linked with the usage of the scarce resources that are available on the WSN. A lifetime of a WSN weighs heavily with the energy resources that are available to it. The problem of wasting valuable amounts of these resources while seeking a time precision that is not required by the application, is critical and has not been clearly addressed by the protocols found on this related work.\\

%-----------------------------------------------------------
% keywords: TTSP architecture
%-----------------------------------------------------------
TTSP provides the capability of fulfilling the synchronization time precision requirements declared by its clients, whether they be the user application running on the nodes or the MAC-layer protocol residing in the operating system of the nodes. The requirements are stated through specific interfaces, which indicate TTSP to keep the synchronization time precision errors below that threshold. Knowing the requested maximum time precision error allowed, TTSP adjusts its synchronization period so that it can minimize the use of its limited resources while keeping every node synchronized with their time precision errors below the requested threshold. TTSP uses a sender-to-receiver scheme for pair-wise synchronization and a master-slave scheme for network-wide synchronization. Several commonly synchronization techniques from these schemes are used. These involve synchronization through message exchange, MAC-layer time-stamping, synchronization in rounds, skew compensation, controlled flooding of broadcasted messages and a master re-election mechanism. To assure that the time precision errors are acceptable within the network, the root node (to which every other node is being synchronized with) is constantly monitoring its synchronizing nodes through their synchronization broadcasts, or in case of a multi-hop, through a forwarded synchronization broadcast. Adjustments to the synchronization period, are taken from these feedback, typically acceptable time precision errors tend to allow a higher synchronization period, while unacceptable precision errors tend to force a decrease of the synchronization period. Thus, the root node adaptively adjusts the synchronization period solely based on the results of its past actions on the synchronization period.\\ 

%-----------------------------------------------------------
% keywords: TTSP implementation
%-----------------------------------------------------------
A reference implementation of TTSP was developed in TinyOS, which is the most popular and open-source operating system available at the moment. TinyOS provides an event-centric and modular architecture, which allowed TTSP implementation development to take benefits from both of these architecture approaches.\\

%-----------------------------------------------------------
% keywords: Evaluation
%-----------------------------------------------------------
Validation of TTSP was conducted in Tagus-SensorNet, a WSN test-bed, which served as a way of assessing TTSP complexity and to collect real-world performance data. The following are some of the most relevant data collected:
\begin{itemize}
\item Regarding the scalability issues when tested with only one and four synchronizing nodes, TTSP tends to converge to a synchronization period faster when more nodes are present in the network, this is clearly justified by the increased quantity of feedback returned from its adjustments. But the increased number of nodes also give room to nodes with completely different clock drift rates than the one from the root node, which in the end, leads to a lower synchronization period, thus a higher resource consumption.
\item A set of time precision requirements were tested, precisely a maximum precision error of 10 ms and 50 ms. Both requirements are fulfilled, although one can observe that for a lower requirement, in this case the 50 ms requirement, the synchronization period convergence seems to be slower than the higher one. This is justified by the intrinsic mechanism of an adaptive approach, in which an adjustment is made and future adjustments will solely depend on the outcome of the received feedback of that previous adjustment. In this particular case, requesting a lower requirement, will let TTSP reach long synchronization periods, which will result in longer feedbacks, thus a slower synchronization period convergence.
\item The fast synchronization mechanism in TTSP allows new arriving or just turned on nodes, to quickly synchronize with the root node after it has already converged to a synchronization period. This mechanism avoids unacceptable waiting times by allowing a new node to synchronize with the root node in less than a minute after it has been detected.
\item Finally, a long run test of TTSP was tested for twenty-four hours, in which the test-bed was subject to all kinds of radio interference which is common in Tagus-SensorNet. The results showed that TTSP guaranteed that the maximum precision error was not reached and that once the synchronization period converged it didn't suffer any changes. This clearly fulfils the objective of adaptively synchronizing Tagus-SensorNet while fulfilling specific time precision requirements.
\end{itemize}

%-----------------------------------------------------------
% keywords: Final conclusion
%-----------------------------------------------------------
The results obtained with TTSP in Tagus-SensorNet make it a valid approach to synchronize clocks in WSNs. By using TTSP for time synchronization in a WSN, it's possible to adapt the time synchronization period to fulfil the time precision requirements while seeking to minimize the resource consumption in the process. TTSP also avoids statically pre-configured synchronization periods which might end not being suitable if for any reason the network topologies changes or nodes get removed or added to the network. TTSP cleverly finds the adequate synchronization period for the requested maximum time precision error allowed in the network, thus avoiding network administrator parametrization. One can also conclude that by using TTSP, it will be possible to maximize the network lifetime, since the most power demanding component of a node, the radio used for message transmission and reception is less used when a better suitable (obtained with TTSP) synchronization period is used. Finally, TTSP accomplishes the initial proposed goals, that is to achieve a network-wide synchronization in a scalable fashion way with requested specific time precision requirements while making an efficient use of the available network resources.

\section{Future Work}

Regarding TTSP, there are some challenges and space for improvement in a future work. The following open issues are but some of the ones that come to mind for further development:
\begin{itemize}
\item TTSP uses a modular architecture, its underlying time synchronization logic can easily be switched as long as it still maintains and makes use of the same interface signatures. Thus, a suggestion would be to adopt other synchronization approach schemes, such as a Receiver-to-receiver scheme for pair-wise synchronization or a Peer-to-peer for network-wide synchronization.
\item Integrating techniques for secure time synchronization in TTSP could prove useful for resilient WSNs.
\item Executing a series of extensive tests in Tagus-Sensornet with different MAC protocols and different time precision requirements or maximum synchronization period.
\item TTSP can easily support node mobility by maintaining a set of samples for each root a node finds during its lifetime, thus allowing multi-root synchronization, this feature has not be tested extensively in Tagus-SensorNet, but should be done for future work, it would also aid in identifying current weaknesses (limitations such as memory size) and elaborate optimizations.
\end{itemize}